\documentclass{article}
\usepackage{fullpage}
\usepackage{biblatex}
\usepackage{hyperref}
\usepackage{nameref}
\usepackage{gantt}
\bibliography{proposal_refs}

\title{Final Year Project Proposal\\ \vspace{0.2cm} \large{A machine learning approach to estimating signal delay caused by gravitational lensing}}
\author{Michal Staniaszek \\\\ 1028907}

\topmargin -0.5in
\headheight 0in
\headsep 0in
\textheight 10.2in

\begin{document}

\maketitle

\subsubsection*{Project details}

The successful completion of the project will require in-depth knowledge of various techniques that can be used to make estimates given a certain data set. It will also require understanding of the gravitational lensing phenomenon, and some of the physics and mathematics behind its workings.

\subsubsection*{Approach}

The project will build on the kernel method approach taken in \cite{t09} and \cite{t06}. As is usual with this type of analysis, it will be necessary to consider observational noise, and also situations where lack of observations leads to gaps in the data. 

\subsubsection*{Timetable}

\begin{gantt}{15}{9}
  \begin{ganttitle}
  \titleelement{2012}{5}
  \titleelement{2013}{4}
  \end{ganttitle}
  \begin{ganttitle}
    \titleelement{Aug}{1}
    \titleelement{Sep}{1}
    \titleelement{Oct}{1}
    \titleelement{Nov}{1}
    \titleelement{Dec}{1}
    \titleelement{Jan}{1}
    \titleelement{Feb}{1}
    \titleelement{Mar}{1}
    \titleelement{Apr}{1}
  \end{ganttitle}
  \ganttbar{Preparation}{0.5}{1.5}
  \ganttbar{Mathematics refresh}{0.5}{2.5}
  \ganttbar{Background reading}{2}{1}
  \ganttbar{Planning}{2}{1}
  \ganttbar{Prototype coding}{2.5}{1.5}
  \ganttbar{Inspection preparation}{3.8}{0.4}
  \ganttmilestone{Inspection Week}{4.2}
  \ganttbar{Coding}{4}{2.5}
  \ganttbar{Experiments}{6}{1}
  \ganttbar{Presentation preparation}{7}{0.66}
  \ganttmilestone{Presentation Week}{7.66}
  \ganttbar{Write dissertation}{6.5}{2}
  \ganttmilestone{Dissertation hand-in}{8.5}
\end{gantt}

\subsubsection*{Hardware \& software}

It is intended that the C programming language be used as the main language for the project, but this is yet to be decided. Other languages in consideration include Haskell, C++ and Python. If the need arises, Matlab may also be used, but it would be preferential to complete the project without its use. There will likely be no hardware requirements for this project other than a system which can run the finished code.

\printbibliography

\end{document}