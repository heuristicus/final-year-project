% $Header: /home/vedranm/bitbucket/beamer/solutions/conference-talks/conference-ornate-20min.en.tex,v 90e850259b8b 2007/01/28 20:48:30 tantau $
\documentclass{beamer}
\usepackage{amsmath,algorithm,algorithmic,tikz,adjustbox,pgfplots,pgfplotstable,subfig,graphicx}
\usetikzlibrary{shapes,arrows,positioning,patterns}
\usepackage[english]{babel}
% % or whatever

\usepackage[latin1]{inputenc}
% % or whatever

%\usepackage[T1]{fontenc}
% Or whatever. Note that the encoding and the font should match. If T1
% does not look nice, try deleting the line with the fontenc.
\usepackage{graphicx}
\usepackage{cmbright}
% This file is a solution template for:

% - Talk at a conference/colloquium.
% - Talk length is about 20min.
% - Style is ornate.



% Copyright 2004 by Till Tantau <tantau@users.sourceforge.net>.
%
% In principle, this file can be redistributed and/or modified under
% the terms of the GNU Public License, version 2.
%
% However, this file is supposed to be a template to be modified
% for your own needs. For this reason, if you use this file as a
% template and not specifically distribute it as part of a another
% package/program, I grant the extra permission to freely copy and
% modify this file as you see fit and even to delete this copyright
% notice. 
\mode<presentation>
{
  \usetheme{Amsterdam}
  %\usefonttheme{serif}
  % or ...

  \setbeamercovered{transparent}
  \beamertemplatenavigationsymbolsempty
  % or whatever (possibly just delete it)
}

\setbeamertemplate{footline}
{
  \leavevmode%
  \hbox{%
  \begin{beamercolorbox}[wd=.333333\paperwidth,ht=2.25ex,dp=1ex,center]{author in head/foot}%
    \usebeamerfont{author in head/foot}\insertshortauthor % Get rid of short institute next to name
  \end{beamercolorbox}%
  \begin{beamercolorbox}[wd=.333333\paperwidth,ht=2.25ex,dp=1ex,center]{title in head/foot}%
    \usebeamerfont{title in head/foot}\insertshorttitle
  \end{beamercolorbox}%
  \begin{beamercolorbox}[wd=.333333\paperwidth,ht=2.25ex,dp=1ex,right]{date in head/foot}%
    \usebeamerfont{date in head/foot}\insertshortdate{}\hspace*{2em}
    \insertframenumber{} / \inserttotalframenumber\hspace*{2ex} 
  \end{beamercolorbox}}%
  \vskip0pt%
}
\makeatother
\setbeamercovered{dynamic}
\title[Time Delay Estimation] % (optional, use only with long paper titles)
{Time Delay Estimation in Gravitationally Lensed Photon Stream Pairs}

\author[Micha{\l} Staniaszek]{\large{Micha{\l} Staniaszek} \\ \scriptsize{Supervisor: Peter Ti\v{n}o}}
\institute[bham]{The University of Birmingham}

\setcounter{tocdepth}{1}

\date{\today}

% If you wish to uncover everything in a step-wise fashion, uncomment
% the following command: 

%\beamerdefaultoverlayspecification{<+->}

\begin{document}

\begin{frame}
  \titlepage
\end{frame}

\begin{frame}{Outline}
  \tableofcontents
  % You might wish to add the option [pausesections]
\end{frame}

\section{The Problem}

\begin{frame}{What is Gravitational Lensing?}
  \begin{itemize}[<+->]
  \item The bending of light due to gravitational effects
  \item Objects such as galaxy clusters affect the path of light
  \item Multiple images of the lensed object can be observed
  \item Source has a characteristic signal
  \item Images have the same signal, but with some time delay $\Delta$
  \end{itemize}
\end{frame}

\begin{frame}{Strong vs. Weak Lensing}
  \begin{block}{Strong Lensing}
    Time delays can be on the order of hundreds of days
      \begin{itemize}
    \item<2-> Daily measurements of photon flux used to observe variation
    \end{itemize}
  \end{block}
  \begin{block}{Weak Lensing}<3->
    \uncover<3->{
    Time delays are on a much shorter timescale
  }
    \begin{itemize}
    \item<4-> Variation in the signal observed on the order of hours rather than days
    \item<5-> Track individual photon arrival times (streams of photons)
    \end{itemize}
  \end{block}
\end{frame}

\section{The Project}

\begin{frame}{Aim of the Project}
  Create a system to estimate the time delay $\Delta$ between pairs of
  photon streams from weakly lensed objects
\end{frame}

\begin{frame}{Motivation}
  \begin{enumerate}
  \item<1-> Form the base for a system to automatically flag potential lensed
    objects
    \uncover<2->{
    \begin{itemize}
    \item<2-> Lots of data, but analysing it all is difficult
    \item<3-> Flag interesting-looking objects for further investigation
    \end{itemize}
  }
  \item<4-> Better estimates of time delay are useful
    \begin{itemize}
    \item<5-> Improved estimates of $H_0$
    \item<6-> Dark matter measurements
    \item<7-> Mass distribution for regions of space
    \end{itemize}
  \end{enumerate}
\end{frame}

\section{System Components}

\begin{frame}
  Three main parts of the system
  \begin{enumerate}
  \item<1-> \alert<4->{Photon stream simulation}
  \item<2-> Function estimation
  \item<3-> Time delay estimation
  \end{enumerate}
\end{frame}

\subsection{Simulating Photons}

\begin{frame}{Photon Simulation}
  We use a nonhomogeneous poisson process to simulate arrival times.
  \begin{itemize}[<+->]
  \item Rate parameter $\lambda$ is the expected number of arrivals per unit time
  \item Waiting time until the next event has an exponential distribution
  \item Time to next event in homogeneous process $t=-\frac{1}{\lambda}\ln(U)$,
    where $U\sim U(0,1)$
  \item Use thinning on events generated using the above to generate times based
    on a nonhomogeneous process
  \end{itemize}
\end{frame}

\begin{frame}{Function Generation}
  To generate events, need some function $\lambda(t)$
  \begin{itemize}[<+->]
  \item Randomly generate function by using Gaussians
  \item Centre Gaussians at uniform intervals $\Delta t$, with standard
    deviation $\alpha \cdot \Delta t$
  \item Sum the Gaussians to give a continuous function
  \end{itemize}
\end{frame}

\begin{frame}[label=gauss]
  \begin{figure}
    \centering
    \begin{tikzpicture}
      \begin{axis}[xmin=0, xmax=100, ymin=-10,
        ymax=25]
        \addplot[mark=none, color=red] table[x index=0,y index=1,col sep=space]
        {random_function_0_sum.dat};
        \addplot[mark=none, color=blue] table[x index=0,y index=1,col sep=space] {random_function_0_contrib.dat};
      \end{axis}
    \end{tikzpicture}
    \caption{The red function is generated by summing the blue Gaussians. Gaussian values are multiplied by 3. Function is shifted so
      all $y\geq 0$}
  \end{figure}
\end{frame}

\subsection{Function Estimation}

\begin{frame}
  Three main parts of the system
  \begin{enumerate}
  \item Photon stream simulation
  \item \alert{Function estimation}
  \item Time delay estimation
  \end{enumerate}
\end{frame}

\begin{frame}{General Idea}
  \begin{enumerate}
  \item Split the interval into bins
  \item Count the number of events that occur in each bin
  \item Estimate functions based on these counts
  \end{enumerate}
\end{frame}

\begin{frame}{Iterative Weighted Least Squares}
  Estimate linear functions of the form $y=a+bx$ using Iterative Weighted Least Squares (IWLS)
  \begin{itemize}[<+->]
  \item Find \[\min_{\alpha,\beta}\sum_{k=1}^{n}w_k\cdot(Y_k-[\alpha+\beta x])^2\]
  \item $\alpha$ and $\beta$ are estimators for $a$ and $b$, $w_k$ is the weight assigned to each value $Y_k$, which is the event count for the $k$th bin. $x$ is the midpoint of the sub-interval.
  \item Update weights at each iteration by using estimated values of $\lambda$ in each sub-interval.
  \end{itemize}
\end{frame}

\begin{frame}{Piecewise}
  Some parts of functions can be reasonably approximated by straight lines
  \begin{itemize}[<+->]
  \item Split the interval into several subintervals and estimate each in turn
  \item Once an estimate is done, extend the line to probe the next interval
  \item If the extension matches the data, keep it
  \end{itemize}
\end{frame}

\begin{frame}{Baseline}
  Characteristic functions of photon streams are continuous - must make the
  piecewise estimate continuous as well.
  \begin{itemize}[<+->]
  \item Modify each interval estimate to make a continuous function
  \item At each breakpoint, find the midpoint between the estimates
  \item Modify function values to make the end point of one interval
    estimate meet the start of the next
  \end{itemize}
\end{frame}

\begin{frame}{Piecewise Estimate Example}
  \begin{center}
    \begin{tikzpicture}
      \pgfplotsset{every axis legend/.append style={
          at={(0.5,1.03)},anchor=south,draw=none, font=\small}}
      \begin{axis}[xmin=0, xmax=100, ymin=0, ymax=25, legend columns=4]
        \addplot[only marks, mark=x, color=green, opacity=0.35] table[x index=0,y index=1,col
        sep=space] {bins.dat};
        \addlegendentry{Bin counts}
        \addplot[mark=none, color=black] table[x index=0,y index=1,col sep=space]
        {real_func.dat};
        \addlegendentry{Real}
        \addplot[mark=none, color=blue] table[x index=0,y index=1,col sep=space]
        {piecewise.dat};
        \addlegendentry{Piecewise}
      \end{axis}
    \end{tikzpicture}
  \end{center}
\end{frame}

\begin{frame}{Baseline Estimate vs Piecewise Estimate}
  \begin{center}
    \begin{tikzpicture}
      \pgfplotsset{every axis legend/.append style={
          at={(0.5,1.03)},anchor=south,draw=none, font=\small}}
      \begin{axis}[xmin=0, xmax=100, ymin=0, ymax=25, legend columns=4]
        \addplot[only marks, mark=x, color=green, opacity=0.35] table[x index=0,y index=1,col
        sep=space] {bins.dat};
        \addlegendentry{Bin counts}
        \addplot[mark=none, color=black] table[x index=0,y index=1,col sep=space]
        {real_func.dat};
        \addlegendentry{Real}
        \addplot[mark=none, color=blue, opacity=0.5] table[x index=0,y index=1,col sep=space]
        {piecewise.dat};
        \addlegendentry{Piecewise}
        \addplot[mark=none, color=red] table[x index=0,y index=1,col sep=space]
        {baseline.dat};
        \addlegendentry{Baseline}
      \end{axis}
    \end{tikzpicture}
  \end{center}
\end{frame}

\begin{frame}{Kernel Density}
  \begin{itemize}[<+->]
  \item Centre a Gaussian kernel at each event time
  \item Sum Gaussians to approximate the function
  \item Must be normalised depending on standard deviation used
  \item Use probability density function to automatically calculate normalisation constant
  \end{itemize}
\end{frame}

%\againframe{gauss}

\subsection{Time Delay Estimation}

\begin{frame}
  Three main parts of the system
  \begin{enumerate}
  \item Photon stream simulation
  \item Function estimation
  \item \alert{Time delay estimation}
  \end{enumerate}
\end{frame}

\begin{frame}{General Idea}
  The actual $\Delta$ is not known, so we make guesses and check to see how good
  they are.
  \begin{itemize}[<+->]
  \item Estimate each stream of photons
  \item Pick a value of $\Delta$ and shift the function estimate
  \item Compare it to the other estimate and see how good the match is
  \item Hierarchical - coarse first pass, improve estimate with finer second pass
  \end{itemize}
\end{frame}

\begin{frame}{Area Between Curves}
  \begin{enumerate}[<+->]
  \item Approximate the area between the two function estimates $\hat{\lambda}_1$
    and $\hat{\lambda}_2$
    \begin{align*}
      d(\hat{\lambda}_1,\hat{\lambda}_2)&=\int(\hat{\lambda}_1(t)-\hat{\lambda}_2(t))^2\,dt\\
      &\approx\frac{1}{N}\sum_{i=1}^N(\hat{\lambda}_1(t)-\hat{\lambda}_2(t))^2
    \end{align*}
  \item Find the value of $\Delta$ for which
    $d(\hat{\lambda}_1,\hat{\lambda}_2)$ is minimised
  \end{enumerate}
\end{frame}

\begin{frame}{Probability Density}
  \begin{enumerate}[<+->]
  \item Pick a value of $\Delta$
  \item Combine function estimates $\hat{\lambda}_1$ and $\hat{\lambda}_2$ into
    an ``average'' function $\overline{\lambda}$, where
    \[\overline{\lambda}(t)=\frac{\hat{\lambda}_1(t)+\hat{\lambda}_2(t+\Delta)}{2}\]
  \item See how well $\overline{\lambda}$ matches the data from the two streams
    by maximising
  \end{enumerate}
  \uncover<3->{\begin{align*}
      \log P(S_A,S_B\mid\overline{\lambda}(t))=\sum_{t=\Delta}^{T-\Delta}&\log P(S_A(t)\mid \overline{\lambda}(t))\\
      &+ \log P(S_B(t+\Delta)\mid \overline{\lambda}(t))
  \end{align*}}
\end{frame}


\section{Experimentation}

\begin{frame}{Experimental Setup}
  Three sets of experiments
  \begin{enumerate}[<+->]
  \item Preliminary sine function experiments
    \begin{itemize}
    \item Vary $\alpha$ in $y=a-b\sin(\alpha t)$
    \item Higher $\alpha$ leads to faster oscillation
    \end{itemize}
  \item Experiments on a smaller range to see degradation
  \item Random functions
    \begin{itemize}
    \item Vary $\alpha$ where standard deviation of Gaussian $\sigma=\alpha\cdot\Delta t$
    \end{itemize}
  \end{enumerate}
\end{frame}

\begin{frame}{Experimental Method}
  \begin{enumerate}[<+->]
  \item Perform model selection on each stream
    \begin{itemize}
    \item withhold some event data
    \item Determine optimal parameters for each set of streams
    \end{itemize}
  \item Estimate time delay for each pair of streams using optimal parameters (with all data)
  \end{enumerate}
\end{frame}

\begin{frame}{Results}
  \begin{itemize}
  \item Area estimator better than PDF, but significance not high
  \item Both types of estimators not significantly different
  \end{itemize}
  \begin{center}
    \begin{figure}
      \centering
      \begin{tabular}{ r|c|c| }
        \multicolumn{1}{r}{}
        &  \multicolumn{1}{c}{Gaussian}
        & \multicolumn{1}{c}{Baseline} \\
        \cline{2-3}
        Area&$15.95\pm 4.51$&$15.99\pm 3.09$\\\cline{2-3}
        PDF&$16.53\pm 11.80$&$15.72\pm 14.06$\\\cline{2-3}
      \end{tabular}
      \caption{Experimental results for $\alpha=0.07$ in the second set of sine
        function experiments ($\mu\pm\sigma$, $n=10$). Actual delay is 15.}
    \end{figure}
  \end{center}
\end{frame}

\begin{frame}{Baseline Estimator Degradation}
  \begin{adjustbox}{max size={.95\textwidth}{.95\textheight}}
    \begin{figure}
      \begin{tabular}{cc}
        \subfloat[Baseline area]{
          \begin{tikzpicture}
            \begin{axis}[xlabel={$\alpha$},ylabel={Error}, scaled x ticks=base 10:2]
              \addplot[only marks,error bars/.cd,y dir=both, y explicit] table[x
              index=0,y index=4,col sep=tab,y error index=3] {baseline_area.txt};
            \end{axis}
          \end{tikzpicture}
        }&\subfloat[Baseline PDF]{
          \begin{tikzpicture}
            \begin{axis}[xlabel={$\alpha$},ylabel={Error}, scaled x ticks=base 10:2]
              \addplot[only marks,error bars/.cd,y dir=both, y explicit] table[x
              index=0,y index=4,col sep=tab,y error index=3] {baseline_pmf.txt};
            \end{axis}
          \end{tikzpicture}
        }\\
      \end{tabular}
    \end{figure}
  \end{adjustbox}
\end{frame}

\begin{frame}{Gaussian Estimator Degradation}
  \begin{adjustbox}{max size={.95\textwidth}{.95\textheight}}
    \begin{figure}
      \begin{tabular}{cc}
        \subfloat[Gaussian area]{
          \begin{tikzpicture}
            \begin{axis}[xlabel={$\alpha$},ylabel={Error}, scaled x ticks=base 10:2]
              \addplot[only marks,error bars/.cd,y dir=both, y explicit] table[x
              index=0,y index=4,col sep=tab,y error index=3] {gaussian_area.txt};
            \end{axis}
          \end{tikzpicture}
        }&\subfloat[Gaussian PDF]{
          \begin{tikzpicture}
            \begin{axis}[xlabel={$\alpha$},ylabel={Error}, scaled x ticks=base 10:2]
              \addplot[only marks,error bars/.cd,y dir=both, y explicit] table[x
              index=0,y index=4,col sep=tab,y error index=3] {gaussian_pmf.txt};
            \end{axis}
          \end{tikzpicture}
        }\\
      \end{tabular}
    \end{figure}
  \end{adjustbox}
\end{frame}

\section{Code Base}

\begin{frame}{System Structure}
  \begin{columns}[t]
    \begin{column}{0.5\textwidth}
      \pgfdeclarelayer{background}
      \pgfdeclarelayer{foreground}
      \pgfsetlayers{background,main,foreground}
      % horizontal separation
      \def \hnsep {0.5}
      \tikzstyle{sub}=[draw, fill=blue!20, text width=5em, 
      text centered, minimum height=2.5em, node distance=1.5cm]
      \begin{adjustbox}{max size={.95\textwidth}{.8\textheight}}
        \begin{tikzpicture}
          \node (param) at (0,3.5) [sub] {Parameter file};
          % libs group
          \node (math) at (2,6) [sub] {Math};
          \node (gut) [sub, right=\hnsep of math] {General};
          \node (file) [sub, right=\hnsep of gut] {File};
          \node (plist) [sub, right=\hnsep of file] {Parameter List};
          \node (lib) [below right=0.25cm and -0.65 of gut] {\textbf{Libraries}};
          % generator group
          \node (hom) at (2,1) [sub] {HPP};
          \node (nhm) [sub, below of=hom] {NHPP};
          \node (rfunc) [sub, below of=nhm] {Random Function};
          \node (gauss) [sub, below of=rfunc] {Gaussian};
          \node (gen) [below of=gauss, font=\small] {\textbf{Generators}};
          \node (strout) [sub, below of=gen] {Stream Data};
          % estimator group
          \node (ln) at (6.17,0) [sub] {Linear};
          \node (pc) [sub, below of=ln] {Piecewise};
          \node (bl) [sub, below of=pc] {Baseline};
          \node (kd) [sub, below of=bl] {Kernel Density};
          \node (td) [sub, below of=kd] {Time Delta};
          \node (est) [below of=td, font=\small] {\textbf{Estimators}};
          \node (estout) [sub, below of=est] {Estimator Output};
          % experimenter
          \node (expparam) at (11.5,2) [sub] {Experiment Parameters};
          \node (exp) at (10,0) [sub] {Harness};
          \node (explbl) [below of=exp, font=\small] {\textbf{Experimenter}};
          \node (expout) [sub, below of=explbl] {Experiment Results};
          % Draw the rest on the background layer
          \begin{pgfonlayer}{background}
            % Estimator background
            \path (ln.north west)+(-0.2,0.2) node (a) {};
            \path (est.south -| ln.east)+(+0.2,-0.2) node (b) {};
            \path[fill=blue!10,rounded corners, draw=black!50, dashed]
            (a) rectangle (b);
            % generator background
            \path (hom.north west)+(-0.2,0.2) node (c) {};
            \path (gen.south -| hom.east)+(+0.2,-0.2) node (d) {};
            \path[fill=blue!10,rounded corners, draw=black!50, dashed]
            (c) rectangle (d);
            % libs background
            \path (math.north west)+(-0.2,0.2) node (e) {};
            \path (lib.south -| plist.east)+(+0.2,-0.2) node (f) {};
            \path[fill=blue!10,rounded corners, draw=black!50, dashed]
            (e) rectangle (f);
            % experimenter background
            \path (exp.north west)+(-0.2,0.2) node (g) {};
            \path (explbl.south -| exp.east)+(+0.2,-0.2) node (h) {};
            \path[fill=blue!10,rounded corners, draw=black!50, dashed]
            (g) rectangle (h);
            
            % path from expparam to experiments
            \coordinate [above=1.49 of exp] (expln) {};
            \coordinate [above=1 of exp] (tpexp) {};
            \draw [dashed,line width=1pt] (expparam.west) -- (expln);
            % path from experiments to exp out
            \draw [->,line width=1pt] (explbl.south)+(0,-0.2) -- (expout.north);

            % library arrows
            \path (ln.north)+(0,0.05) node (esttop){};    
            \coordinate [above=0.2cm of hom] (gentop) {};
            \coordinate [below=2cm of lib] (lsplit) {};
            \coordinate [below=0.2cm of lib] (blwlib) {};
            \coordinate [above=0.2cm of exp] (abvexp) {};
            \draw [-,line width=1pt] (blwlib) -- (lsplit);
            \draw [->,line width=1pt] (lsplit) -- (esttop);
            \draw [->,line width=1pt] (lsplit) -| (abvexp);
            \draw [->,line width=1pt] (lsplit) -| (gentop);

            % path from param to library link
            \coordinate [above=0.8cm of lsplit] (tt) {};
            \draw [dashed,line width=1pt] (param.east) -- (tt);
            
            % estimator arrows
            \draw [->] (ln.south) -- (pc.north);
            \draw [->] (pc.south) -- (bl.north);
            \coordinate [below=0.2 of est] (blest) {};
            \draw [->,line width=1pt] (blest)--(estout);
            \coordinate [right=0.9 of estout] (restout) {};
            \draw [dashed,line width=1pt] (estout.east) -- (restout);
            \draw [dashed,line width=1pt] (restout) |- (tpexp);
            
            % generator arrows
            \coordinate [above= 1 of ln] (abvln) {}; %above length est
            \coordinate [below=0.2 of gen] (bgen) {};
            \coordinate [right=0.9 of strout] (rstrout) {};
            \draw [->,line width=1pt] (bgen) -- (strout);
            \draw [->] (hom.south) -- (nhm.north);
            \draw [dashed,line width=1pt] (strout.east) -- (rstrout);
            \draw [dashed,line width=1pt] (rstrout) |- (abvln);
            
          \end{pgfonlayer}
          % \node (lib) at (0,0) [sub] {Libraries};
          % \node (est) at (1.5,-1) [sub] {Estimators};
          % \node (gen) at (-1.5,-1) [sub] {Generators};
          % \draw [->] (lib.east) -| (est.north);
          % \draw [->] (lib.west) -| (gen.north);
        \end{tikzpicture}
      \end{adjustbox}
    \end{column}
    \begin{column}{0.5\textwidth}
      \begin{itemize}
      \item Modular structure
      \item Shared libraries for common functions
      \item Parameter files control behaviour
      \item Command line interface
      \item 7000 lines of C code
      \item Testing using Check utility
      \item Uses Automake
      \item In the style of a GNU package
      \end{itemize}
    \end{column}
  \end{columns}
 \end{frame}

\section*{Summary}

\begin{frame}{Summary}

  % Keep the summary *very short*.
  \begin{itemize}[<+->]
  \item We want to find the value of $\Delta$, the time delay between
    characteristic functions of photon streams
  \item Photon stream simulation using Poisson processes
  \item Estimation of characteristic function of stream using baseline or kernel density estimators
  \item Estimation of time delay with PDF or area estimators
  \item Experimental results indicate area estimator is better than PDF, but
    significance is not high
  \end{itemize}
\end{frame}

\end{document}


