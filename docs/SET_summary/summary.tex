% Created 2013-06-30 Sun 13:52
\documentclass[11pt]{article}
\usepackage[utf8]{inputenc}
\usepackage[T1]{fontenc}
\usepackage{graphicx}
\usepackage{longtable}
\usepackage{float}
\usepackage{wrapfig}
\usepackage{soul}
\usepackage{amssymb}
\usepackage{hyperref}
\usepackage{fontspec}
\usepackage[titletoc,page,title]{appendix}
\usepackage{biblatex}
\usepackage{metalogo}
\usepackage{graphicx}
\usepackage{moreverb}
\usepackage{fancyvrb}
\usepackage{fullpage}
\usepackage{setspace}
\usepackage{subfig}
\usepackage{algorithm}
\usepackage{algorithmic}
\usepackage[scientific-notation=true]{siunitx}
\usepackage{float}
\let\iint\relax % otherwise errors are thrown by amsmath. Defined in latexsym
\let\iiint\relax
\usepackage{amsmath}
\usepackage{hyperref}
\usepackage{tikz}
\usetikzlibrary{positioning}
\bibliography{summary}
\defaultfontfeatures{Mapping=tex-text}
\setromanfont[Ligatures={Common},Numbers={Lining}]{Linux Libertine}

\title{Time Delay Estimation in Gravitationally Lensed Photon Stream Pairs}
\author{\Large{Micha{\l} Staniaszek} \\\small{Supervisor: Peter Ti{\v{n}}o}}
\date{\today}

\begin{document}

\maketitle


this is the abstract

\section{Introduction}
\label{sec-1}

\begin{itemize}
\item explain the project in layman's terms
\end{itemize}
\section{Background}
\label{sec-2}

\begin{itemize}
\item Ideas behind the project
\item what it's useful for
\item what gravitational lensing and time delay are
\end{itemize}
\section{Photon Stream Simulation}
\label{sec-3}

In the early stages of the project, we developed a subsystem which could be used
to generate simulated photon stream data to use for the development and testing
of the rest of the project. The only property of the photons which we are
interested in is their arrival time at our capture device, so the simulator
should produce some event vector $\Phi=\left[\phi_0,\dots,\phi_N\right], \phi_n
\in \mathbb{R}$, where $\phi_n$ is the arrival time of the $n\text{th}$
photon. In order to generate arrival times, we represent the source as some
random variable $X$, which defines the average number of photons per unit time
that arrive at the capture device, and whose varies according to the
characteristic function of the source object.

The characteristic function of $X$ is modelled as a non-homogeneous Poisson
process (NHPP) with continuous function of time, $\lambda(t)$, known as the rate
function. The rate function can be specified either by providing an expression
which is a function of $t$, or by sampling from a randomly generated
function. Random functions are constructed by uniformly distributing $M$
Gaussians across the interval $\left[t_0,T\right]$ in which arrival times are to
be generated. Each Gaussian $g_i$ is defined by its mean $\mu$$_i$, its width
$\sigma$$_i$, and its weight $w_i$, which determines its height. The means of
successive Gaussians are separated by some distance $\Delta t$, such that
$\mu_{m+1}=\mu_m + \Delta t,\text{ where } \mu_0=0$. Greater variation in the
functions is introduced by sampling the weights $w_i$ from a uniform
distribution $U(-1,1)$ and scaling them by some multiplier. The value of the
randomly generated function at some time $t$ is computed by a weighted sum of
Gaussians.

\begin{align}
   \lambda(t) = \sum_{i=0}^M w_i\cdot e^{-(t-\mu_i)^2/2\sigma_i^2}
\end{align}

Having defined or constructed $\lambda(t)$, photon arrival times are generated
from a homogeneous Poisson process (HPP) with constant rate $\lambda$, using
inverse transform sampling. The waiting time to the next event in a Poisson
process is \cite{1998art}
\begin{align}\label{eq:homlambda}
t=-\frac{1}{\lambda}\log(U)
\end{align} where $U\sim U(0,1)$. Knowing this, it is possible to generate
successive events of a HPP for any finite interval, from which events for the
NHPP can then be extracted by thinning, using Algorithm \ref{alg:seq}. The
number of events added to the event vector $\Phi$ in any given interval is
proportional to the value of $\lambda(t)$ in that interval; the probability of
adding an event is low when $\lambda(t)$ is small, and increases with the
value of the rate function.

\begin{algorithm}[H]
\begin{algorithmic}[1]
\REQUIRE $\lambda\geq \lambda(t), t_0 \leq t \leq T$
\STATE $\Phi=\emptyset$, $t=t_0$, $T=\text{interval length}$
\WHILE{$t<T$}
\STATE Generate $U_1\sim U(0,1)$
\STATE $t=t-\frac{1}{\lambda}\ln(U_1)$
\STATE Generate $U_2\sim U(0,1)$, independent of $U_1$
\IF{$U_2\leq\frac{\lambda(t)}{\lambda}$}
\STATE $\Phi \leftarrow t$
\ENDIF
\ENDWHILE
\RETURN $\Phi$
\end{algorithmic}
\caption{Generating event times for a NHPP by thinning}
\label{alg:seq}
\end{algorithm}

\section{Function Estimation}
\label{sec-4}

The function estimator subsystem receives input of the event vector $\Phi$, and
attempts to reconstruct the rate function. Due to the nature of the data, it is
not possible to find the true rate function---only an estimate is possible.
\subsection{Baseline Estimation}
\label{sec-4.1}

    As mentioned in the previous section, the piecewise IWLS estimator gives us
    a piecewise disjoint estimate of the function, but we would like one which
    is piecewise continuous. In order to do this, the end of each interval
    estimate must meet the start of the next. The estimate returned by the
    piecewise estimator has several breakpoints---points where the start of one
    sub-interval and the end of another meet. If there are $L$ lines that make
    up the estimate, there will be $R=L-1$ breakpoints. At each of these
    breakpoints $r$, we calculate the value of the previous and subsequent
    function estimates $f$, and find their midpoint $m$ with
    \begin{equation}
    m_i = \frac{f_{i}(r_i) + f_{i+1}(r_i)}{2},\quad 0\leq i < R
    \end{equation}
    The value of $m$ is calculated for each breakpoint. Midpoints are not
    calculated at time 0 and time $T$. Instead, the function values at those
    points are used. Each sub-interval is now represented by a point $p$ at the
    start and $q$ at the end, each with an $x$ and $y$ coordinate. With these
    points, we can recalculate each sub-interval estimate $f$ of the form
    $y=\hat{a}+\hat{b}x$ by replacing $y$ with $p_y$ and $x$ with $p_x$, and
    recalculating the gradient $\hat{b}$ and intercept $\hat{a}$ with
    \begin{align}
    \hat{b} &= \frac{q_y-p_y}{q_x-p_x}\\
    \hat{a} &= p_y - \hat{b}\cdot p_x
    \end{align}
    In this way, each sub-interval estimate links points $p$ and $q$, giving us
    a piecewise continuous function estimate, and this step completes the first
    function estimation method. Figure \ref{fig:basecomp} shows an example of a
    piecewise and baseline estimate.
\subsection{Kernel Density Estimation}
\label{sec-4.2}

   The second function estimation method implemented was a kernel density
   estimator, which uses \emph{kernels} to estimate the probability density of a
   random variable. A kernel is simply a weighting function, which affects how
   much a given sample is considered when constructing the function
   estimate. Since the photon stream data is assumed to be generated by a source
   whose variability is defined by some random variable, the event times are a
   sample drawn from the PDF of that variable. We use a Gaussian kernel
   \begin{align}
   K(t,\mu)=e^{-(t-\mu)^2/2\sigma^2}
   \end{align}
   to estimate the PDF, centring a kernel at each photon arrival time $\phi_n$ by
   setting $\mu=\phi_n$. The width of the kernel depends on some fixed value
   $\sigma$. We perform a Gauss transform on the $N$ kernels, finding the
   contribution of all the kernels at $M$ points in time, from which we get an
   estimate $\hat{\lambda}(t)$ of the characteristic function.
   \begin{align}
   \hat{\lambda}(t_i) = \sum_{j=1}^N K(t_i,\mu_j), \quad i=1,\dots,M
   \end{align}
   Using a larger $M$ gives a higher resolution. Depending on the value of
   $\sigma$ used, $\hat{\lambda}(t)$ will be some multiple of the actual
   function $\lambda(t)$. Thus, the final step is to normalise
   $\hat{\lambda}(t)$. We split the stream data into $B$ bins with midpoints $b$
   and calculate the bin count $x$ for each. We start with the normalisation
   constant $\eta$ at a low value, and gradually increase it to some threshold,
   finding
   \begin{equation}\label{eq:normcalc}
   \sum_{i=1}^B
   \log\left(\frac{\phi^xe^{-\phi}}{x!}\right), \quad \phi=\eta\cdot\hat{\lambda}(b_i)
   \end{equation}
   for each value of $\eta$. The value of $\eta$ which maximises this sum of log
   Poisson PDFs is used to normalise $\hat{\lambda}(t)$ in subsequent
   computations. Figure \ref{fig:kde} shows an example of a kernel density
   estimate, and displays a weakness in the estimator. As one moves towards the
   start or end of the interval, fewer Gaussians make a noticeable contribution
   to the function calculation, resulting in a drop-off of the estimate.
\section{Time Delay Estimation}
\label{sec-5}

  Once we are able to estimate the characteristic function of photon streams, we
  can use these estimates to compute an estimate of the time delay between two
  streams. If the two streams come from the same source, then they should have
  the same characteristic function, but delayed by some value $\Delta$. Our
  estimates of the characteristic function will differ for both streams due to
  the fact that the number of photon arrivals in each bin will be different for
  each stream, but each should look relatively similar. In this section we
  present two methods for estimating the time delay between a pair of streams
  based on their function estimates.

  Both of the estimators work by starting $\Delta$ at $-\Delta_{\text{max}}$,
  and increment it by some step until reach $+\Delta_{\text{max}}$ is reached,
  using a metric to evaluate how good the estimate is with that value. It is
  important to note that the value of $\Delta_{\text{max}}$ defines the interval
  in which the metric is computed. The need for calculation only in some
  specific interval should be clear---if one function is shifted by $\Delta$,
  and both functions have the same time interval, then there will be an interval
  of length $\Delta$ at either end of the range in which only one of the
  function estimates has values. As such, the metric can only be computed in the
  overlapping area. Varying $\Delta$ changes the overlapping interval. Setting
  $\Delta=0$ minimises the value, and $\Delta=\pm\Delta_{\text{max}}$ maximises
  it. Performing calculations on different interval lengths would require the
  value of the metric for longer intervals to be scaled to that of the
  shortest. To make useful comparisons, we must perform calculations only on the
  interval in which the two functions overlap for all values of
  $\Delta$. Imposing this constraint means that the value of
  $\Delta_{\text{max}}$ can never exceed the interval length $T_{\text{est}}$ in which we are
  performing the estimate. We are left with the constraints
  $T_{\text{est}}=[t_0+\Delta_{\text{max}},
  T-\Delta_{\text{max}}],\,\Delta_{\text{max}}<T$ on the interval and the
  maximum value of $\Delta$.
\subsection{Area Method}
\label{sec-5.1}

   The first of the two methods uses a very simple metric to estimate the time
   delay. By taking the two function estimates, we can attempt to match up the
   two functions so that they ``fit together'' best. The goodness of fit can be
   determined by the area between the two functions $\hat{\lambda}_1$ and
   $\hat{\lambda}_2$, calculated by
   \begin{align}
   \begin{split}
   d(\hat{\lambda}_1,\hat{\lambda}_2)&=\int(\hat{\lambda}_1(t)-\hat{\lambda}_2(t+\Delta))^2\,dt\\
   &\approx\frac{1}{N}\sum_{i=1}^N(\hat{\lambda}_1(t)-\hat{\lambda}_2(t+\Delta))^2
   \end{split}
   \end{align}
   for each value of $\Delta$. Our estimate of $\Delta$ is set to the value at
   which $d(\hat{\lambda}_1,\hat{\lambda}_2)$ is minimised. Rather than using an
   integral to get the exact area between the functions, we use a less
   computationally expensive discrete approximation.
\subsection{PDF Method}
\label{sec-5.2}

   The second method of estimation is using probability density functions. As
   before, we guess a value of $\Delta$ between $-\Delta_{\text{max}}$ and
   $+\Delta_{\text{max}}$ and shift $\hat{\lambda}_2$ by that amount. However,
   we know that there must be a single characteristic function, and we want to
   see how well our estimate of that matches the bin counts in each stream. We
   make an ``average'' function $\bar{\lambda}$ by combining the two function
   estimates we have, $\hat{\lambda}_1$ and $\hat{\lambda}_2$ (which is shifted
   by $\Delta$).
   \begin{equation}
   \bar{\lambda}(t)=\frac{\hat{\lambda}_1(t)+\hat{\lambda}_2(t+\Delta)}{2}
   \end{equation}
   The point on $\bar{\lambda}$ at time $t$ is the midpoint between the values of
   the two estimates at that time. Once we have $\bar{\lambda}$, we can assign some
   score to the current estimate of the value of $\Delta$.
   \begin{align}
   \begin{split}
   \log P(S_A,S_B\mid\bar{\lambda}(t))=\sum_{t=\Delta_{\text{max}}}^{T-\Delta_{\text{max}}}&\log P(S_A(t)\mid \bar{\lambda}(t))\\
   &+ \log P(S_B(t+\Delta)\mid \bar{\lambda}(t))\\
   \end{split}
   \end{align}
   Here, we calculate the probability that the function $\bar{\lambda}$ is the
   characteristic function of the two streams $S_A$ and $S_B$. The streams are
   split into bins, and the log probability of the number of events in each bin
   given the value of $\lambda$ calculated for that bin is computed and summed
   over all bins, as in Equation \eqref{eq:normcalc}.

   The calculation of $\lambda$ is slightly more complicated than just taking
   its value at the midpoint of each bin. Since we are considering a number of
   events occurring in a given interval, we must consider the value of $\lambda$
   for the same interval. In order to do this, we use a discrete approximation
   of integrating $\lambda(t)$ over the interval.
   \begin{align}
   \lambda_{a,b}&=\int_a^b\lambda(t)\,dt
   \end{align}
   In the approximation $t$ is incremented by some finite step for each
   successive value. The smaller the value of the step the more accurate the
   approximation of $\lambda_{a,b}$ becomes. As with the previous estimator, the
   estimate is made in two stages, first with a coarse pass over the values of
   delta to compute an initial estimate, and then a finer second pass around the
   first estimated value in order to refine the estimate. 
\section{Experimental Results}
\label{sec-6}

\begin{itemize}
\item general explanation of the experiments performed
\item how was model selection done
\item what sort of data were experiments performed on
\end{itemize}
\section{System}
\label{sec-7}

\begin{itemize}
\item very brief explanation of the system features
\end{itemize}
\section{Conclusion}
\label{sec-8}

\begin{itemize}
\item some suggestions for extensions
\end{itemize}
\printbibliography

$^{1}$ FOOTNOTE DEFINITION NOT FOUND: 1


\end{document}